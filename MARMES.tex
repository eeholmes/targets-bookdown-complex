% Options for packages loaded elsewhere
\PassOptionsToPackage{unicode}{hyperref}
\PassOptionsToPackage{hyphens}{url}
%
\documentclass[
  12pt,
]{book}
\usepackage{lmodern}
\usepackage{amsmath}
\usepackage{ifxetex,ifluatex}
\ifnum 0\ifxetex 1\fi\ifluatex 1\fi=0 % if pdftex
  \usepackage[T1]{fontenc}
  \usepackage[utf8]{inputenc}
  \usepackage{textcomp} % provide euro and other symbols
  \usepackage{amssymb}
\else % if luatex or xetex
  \usepackage{unicode-math}
  \defaultfontfeatures{Scale=MatchLowercase}
  \defaultfontfeatures[\rmfamily]{Ligatures=TeX,Scale=1}
\fi
% Use upquote if available, for straight quotes in verbatim environments
\IfFileExists{upquote.sty}{\usepackage{upquote}}{}
\IfFileExists{microtype.sty}{% use microtype if available
  \usepackage[]{microtype}
  \UseMicrotypeSet[protrusion]{basicmath} % disable protrusion for tt fonts
}{}
\makeatletter
\@ifundefined{KOMAClassName}{% if non-KOMA class
  \IfFileExists{parskip.sty}{%
    \usepackage{parskip}
  }{% else
    \setlength{\parindent}{0pt}
    \setlength{\parskip}{6pt plus 2pt minus 1pt}}
}{% if KOMA class
  \KOMAoptions{parskip=half}}
\makeatother
\usepackage{xcolor}
\IfFileExists{xurl.sty}{\usepackage{xurl}}{} % add URL line breaks if available
\IfFileExists{bookmark.sty}{\usepackage{bookmark}}{\usepackage{hyperref}}
\hypersetup{
  pdftitle={Multivariate Autoregressive Modeling for the Environmental Sciences},
  pdfauthor={E. E. Holmes, M. D. Scheuerell, and E. J. Ward},
  hidelinks,
  pdfcreator={LaTeX via pandoc}}
\urlstyle{same} % disable monospaced font for URLs
\usepackage{color}
\usepackage{fancyvrb}
\newcommand{\VerbBar}{|}
\newcommand{\VERB}{\Verb[commandchars=\\\{\}]}
\DefineVerbatimEnvironment{Highlighting}{Verbatim}{commandchars=\\\{\}}
% Add ',fontsize=\small' for more characters per line
\usepackage{framed}
\definecolor{shadecolor}{RGB}{248,248,248}
\newenvironment{Shaded}{\begin{snugshade}}{\end{snugshade}}
\newcommand{\AlertTok}[1]{\textcolor[rgb]{0.94,0.16,0.16}{#1}}
\newcommand{\AnnotationTok}[1]{\textcolor[rgb]{0.56,0.35,0.01}{\textbf{\textit{#1}}}}
\newcommand{\AttributeTok}[1]{\textcolor[rgb]{0.77,0.63,0.00}{#1}}
\newcommand{\BaseNTok}[1]{\textcolor[rgb]{0.00,0.00,0.81}{#1}}
\newcommand{\BuiltInTok}[1]{#1}
\newcommand{\CharTok}[1]{\textcolor[rgb]{0.31,0.60,0.02}{#1}}
\newcommand{\CommentTok}[1]{\textcolor[rgb]{0.56,0.35,0.01}{\textit{#1}}}
\newcommand{\CommentVarTok}[1]{\textcolor[rgb]{0.56,0.35,0.01}{\textbf{\textit{#1}}}}
\newcommand{\ConstantTok}[1]{\textcolor[rgb]{0.00,0.00,0.00}{#1}}
\newcommand{\ControlFlowTok}[1]{\textcolor[rgb]{0.13,0.29,0.53}{\textbf{#1}}}
\newcommand{\DataTypeTok}[1]{\textcolor[rgb]{0.13,0.29,0.53}{#1}}
\newcommand{\DecValTok}[1]{\textcolor[rgb]{0.00,0.00,0.81}{#1}}
\newcommand{\DocumentationTok}[1]{\textcolor[rgb]{0.56,0.35,0.01}{\textbf{\textit{#1}}}}
\newcommand{\ErrorTok}[1]{\textcolor[rgb]{0.64,0.00,0.00}{\textbf{#1}}}
\newcommand{\ExtensionTok}[1]{#1}
\newcommand{\FloatTok}[1]{\textcolor[rgb]{0.00,0.00,0.81}{#1}}
\newcommand{\FunctionTok}[1]{\textcolor[rgb]{0.00,0.00,0.00}{#1}}
\newcommand{\ImportTok}[1]{#1}
\newcommand{\InformationTok}[1]{\textcolor[rgb]{0.56,0.35,0.01}{\textbf{\textit{#1}}}}
\newcommand{\KeywordTok}[1]{\textcolor[rgb]{0.13,0.29,0.53}{\textbf{#1}}}
\newcommand{\NormalTok}[1]{#1}
\newcommand{\OperatorTok}[1]{\textcolor[rgb]{0.81,0.36,0.00}{\textbf{#1}}}
\newcommand{\OtherTok}[1]{\textcolor[rgb]{0.56,0.35,0.01}{#1}}
\newcommand{\PreprocessorTok}[1]{\textcolor[rgb]{0.56,0.35,0.01}{\textit{#1}}}
\newcommand{\RegionMarkerTok}[1]{#1}
\newcommand{\SpecialCharTok}[1]{\textcolor[rgb]{0.00,0.00,0.00}{#1}}
\newcommand{\SpecialStringTok}[1]{\textcolor[rgb]{0.31,0.60,0.02}{#1}}
\newcommand{\StringTok}[1]{\textcolor[rgb]{0.31,0.60,0.02}{#1}}
\newcommand{\VariableTok}[1]{\textcolor[rgb]{0.00,0.00,0.00}{#1}}
\newcommand{\VerbatimStringTok}[1]{\textcolor[rgb]{0.31,0.60,0.02}{#1}}
\newcommand{\WarningTok}[1]{\textcolor[rgb]{0.56,0.35,0.01}{\textbf{\textit{#1}}}}
\usepackage{longtable,booktabs}
\usepackage{calc} % for calculating minipage widths
% Correct order of tables after \paragraph or \subparagraph
\usepackage{etoolbox}
\makeatletter
\patchcmd\longtable{\par}{\if@noskipsec\mbox{}\fi\par}{}{}
\makeatother
% Allow footnotes in longtable head/foot
\IfFileExists{footnotehyper.sty}{\usepackage{footnotehyper}}{\usepackage{footnote}}
\makesavenoteenv{longtable}
\usepackage{graphicx}
\makeatletter
\def\maxwidth{\ifdim\Gin@nat@width>\linewidth\linewidth\else\Gin@nat@width\fi}
\def\maxheight{\ifdim\Gin@nat@height>\textheight\textheight\else\Gin@nat@height\fi}
\makeatother
% Scale images if necessary, so that they will not overflow the page
% margins by default, and it is still possible to overwrite the defaults
% using explicit options in \includegraphics[width, height, ...]{}
\setkeys{Gin}{width=\maxwidth,height=\maxheight,keepaspectratio}
% Set default figure placement to htbp
\makeatletter
\def\fps@figure{htbp}
\makeatother
\setlength{\emergencystretch}{3em} % prevent overfull lines
\providecommand{\tightlist}{%
  \setlength{\itemsep}{0pt}\setlength{\parskip}{0pt}}
\setcounter{secnumdepth}{5}
\usepackage{booktabs}
\usepackage{amsthm}
\makeatletter
\def\thm@space@setup{%
  \thm@preskip=8pt plus 2pt minus 4pt
  \thm@postskip=\thm@preskip
}
\makeatother
\setcounter{MaxMatrixCols}{20}
\ifluatex
  \usepackage{selnolig}  % disable illegal ligatures
\fi
\usepackage[]{natbib}
\bibliographystyle{apalike}

\title{Multivariate Autoregressive Modeling for the Environmental Sciences}
\author{E. E. Holmes, M. D. Scheuerell, and E. J. Ward}
\date{2022-02-20}

\begin{document}
\maketitle

{
\setcounter{tocdepth}{1}
\tableofcontents
}
\hypertarget{preface}{%
\chapter*{Preface}\label{preface}}


This is material that was developed as part of a course we teach at the University of Washington on applied time series analysis for fisheries and environmental data. You can find our lectures on our course website \href{https://nwfsc-timeseries.github.io/atsa/}{ATSA}.

\hypertarget{book-package}{%
\subsection*{Book package}\label{book-package}}


The book uses a number of R packages and a variety of fisheries data sets. The packages and data sets can be installed by installing our \textbf{atsalibrary} package which is hosted on GitHub:

\begin{Shaded}
\begin{Highlighting}[]
\FunctionTok{library}\NormalTok{(devtools)}
\NormalTok{devtools}\SpecialCharTok{::}\FunctionTok{install\_github}\NormalTok{(}\StringTok{"nwfsc{-}timeseries/atsalibrary"}\NormalTok{)}
\end{Highlighting}
\end{Shaded}

\hypertarget{authors}{%
\subsection*{Authors}\label{authors}}


The authors are United States federal research scientists. This work was conducted as part of our jobs at the Northwest Fisheries Science Center (NWFSC), a research center for NOAA Fisheries, and the United States Geological Survey, which are United States federal government agencies. E. Holmes and E. Ward are affiliate faculty and M. Scheuerell is an associate professor at the University of Washington.

Links to more code and publications can be found on our academic websites:

\begin{itemize}
\tightlist
\item
  \url{http://faculty.washington.edu/eeholmes}
\item
  \url{http://faculty.washington.edu/scheuerl}
\item
  \url{http://faculty.washington.edu/warde}
\end{itemize}

\hypertarget{citation}{%
\subsection*{Citation}\label{citation}}


Holmes, E. E., M. D. Scheuerell, and E. J. Ward. Multivariate Autoregressive Modeling for the Environmental Sciences. NOAA Fisheries, Northwest Fisheries Science Center, 2725 Montlake Blvd E., Seattle, WA 98112. Contacts \href{mailto:eeholmes@uw.edu}{\nolinkurl{eeholmes@uw.edu}}, \href{mailto:eward@uw.edu}{\nolinkurl{eward@uw.edu}}, and \href{mailto:scheuerl@uw.edu}{\nolinkurl{scheuerl@uw.edu}}.

\hypertarget{preface-1}{%
\chapter*{Preface}\label{preface-1}}


\hypertarget{template}{%
\chapter{Template}\label{template}}

Start with one paragraph overview.

A script with all the R code in the chapter can be downloaded
\href{./Rcode/template.R}{here}. The Rmd for this chapter can be downloaded
\href{./Rmds/template.Rmd}{here}.

\hypertarget{data-and-packages}{%
\subsection{Data and packages}\label{data-and-packages}}

All the data used in the chapter are in the \textbf{MARSS} package. Install
the package, if needed, and load to run the code in the chapter.

\begin{verbatim}
library(MARSS)
\end{verbatim}

\hypertarget{introduction}{%
\section{Introduction}\label{introduction}}

Introdution. Here's how to do an equation with numbering.

\hypertarget{new-section}{%
\section{New section}\label{new-section}}

New section. Example of an equation with matrix.

and the process model would look like

The observation errors would be

And the process errors would be

\hypertarget{section-3}{%
\section{Section 3}\label{section-3}}

Another section.

\hypertarget{discussion}{%
\section{Discussion}\label{discussion}}

For your homework this week, we will continue to investigate common
trends in the Lake Washington plankton data.

\begin{enumerate}
\def\labelenumi{\arabic{enumi}.}
\item
  Fit other DFA models to the phytoplankton data with varying numbers
  of trends from 1-4 (we fit a 3-trend model above). Do not include
  any covariates in these models. Using \texttt{R="diagonal\ and\ unequal"} for
  the observation errors, which of the DFA models has the most support
  from the data?

  Plot the model states and loadings as in Section
  \citet{ref}(sec-dfa-estimated-states). Describe the general patterns in the
  states and the ways the different taxa load onto those trends.

  Also plot the the model fits as in Section \citet{ref}(sec-dfa-plot-data).
  Do they reasonable? Are there any particular problems or outliers?
\item
  How does the best model from Question 1 compare to a DFA model with
  the same number of trends, but with \texttt{R="unconstrained"}?

  Plot the model states and loadings as in Section
  \citet{ref}(sec-dfa-estimated-states). Describe the general patterns in the
  states and the ways the different taxa load onto those trends.

  Also plot the the model fits as in Section \citet{ref}(sec-dfa-plot-data).
  Do they reasonable? Are there any particular problems or outliers?
\item
  Fit a DFA model that includes temperature as a covariate and 3
  trends (as in Section \citet{ref}(sec-dfa-lakeWA)), but
  with\texttt{R="unconstrained"}? How does this model compare to the model
  with \texttt{R="diagonal\ and\ unequal"}? How does it compare to the model in
  Question 2?

  Plot the model states and loadings as in Section
  \citet{ref}(sec-dfa-estimated-states). Describe the general patterns in the
  states and the ways the different taxa load onto those trends.

  Also plot the the model fits as in Section \citet{ref}(sec-dfa-plot-data).
  Do they reasonable? Are there any particular problems or outliers?
\end{enumerate}

\textless!DOCTYPE html\textgreater{}

chap02-uss.knit

\hypertarget{header}{}

\hypertarget{sec-uss-head}{}
Univariate models

Start with one paragraph overview.

A script with all the R code in the chapter can be downloaded here. The Rmd for this chapter can be downloaded here.

\hypertarget{data-and-packages}{}
Data and packages

All the data used in the chapter are in the MARSS package. Install the package, if needed, and load to run the code in the chapter.

\hypertarget{sec-uss-intro}{}
Introduction

Introdution. Here's how to do an equation with numbering.

{\[\begin{equation}
\begin{gathered}
\mathbf{y}_t = \mathbf{Z}\mathbf{x}_t+\mathbf{a}+\mathbf{v}_t \text{ where } \mathbf{v}_t \sim \text{MVN}(0,\mathbf{R}) \\
\mathbf{x}_t = \mathbf{x}_{t-1}+\mathbf{w}_t \text{ where } \mathbf{w}_t \sim \text{MVN}(0,\mathbf{Q}) \\
\end{gathered}
\label{eq:template1}
\end{equation}\]}

\hypertarget{sec-uss-2}{}
New section

New section. Example of an equation with matrix.

{\[\begin{equation}
\begin{bmatrix}
    y_{1} \\
    y_{2} \\
    y_{3} \\
    y_{4} \\
    y_{5} \end{bmatrix}_t = 
 \begin{bmatrix}
    z_{11}&amp;z_{12}&amp;z_{13}\\
    z_{21}&amp;z_{22}&amp;z_{23}\\
    z_{31}&amp;z_{32}&amp;z_{33}\\
    z_{41}&amp;z_{42}&amp;z_{43}\\
    z_{51}&amp;z_{52}&amp;z_{53}\end{bmatrix}   
 \begin{bmatrix}
    x_{1} \\
    x_{2} \\
    x_{3} \end{bmatrix}_t + 
 \begin{bmatrix}
    a_1 \\
    a_2 \\
    a_3 \\
    a_4 \\
    a_5 \end{bmatrix} + 
 \begin{bmatrix}
    v_{1} \\
    v_{2} \\
    v_{3} \\
    v_{4} \\
    v_{5} \end{bmatrix}_t.
\label{eq:template2-y}
\end{equation}\]}

and the process model would look like

{\[\begin{equation}
\begin{bmatrix}
    x_{1} \\
    x_{2} \\
    x_{3} \end{bmatrix}_t = 
 \begin{bmatrix}
    1&amp;0&amp;0 \\
    0&amp;1&amp;0 \\
    0&amp;0&amp;1 \end{bmatrix} 
 \begin{bmatrix}
    x_{1}\\
    x_{2}\\
    x_{3}\end{bmatrix}_{t-1} +
 \begin{bmatrix}
    w_{1} \\
    w_{2} \\
    w_{3} \end{bmatrix}_t 
\label{eq:tempate2-x}
\end{equation}\]}

The observation errors would be

{\[\begin{equation}
\begin{bmatrix}
    v_{1} \\
    v_{2} \\
    v_{3} \\
    v_{4} \\
    v_{5} \end{bmatrix}_t 
 \sim \text{MVN} \begin{pmatrix}
    \begin{bmatrix}
    0 \\
    0 \\
    0 \\
    0 \\
    0 \end{bmatrix},
 \begin{bmatrix}
    r_{11}&amp;r_{12}&amp;r_{13}&amp;r_{14}&amp;r_{15}\\
    r_{12}&amp;r_{22}&amp;r_{23}&amp;r_{24}&amp;r_{25}\\
    r_{13}&amp;r_{23}&amp;r_{33}&amp;r_{34}&amp;r_{35}\\
    r_{14}&amp;r_{24}&amp;r_{34}&amp;r_{44}&amp;r_{45}\\
    r_{15}&amp;r_{25}&amp;r_{35}&amp;r_{45}&amp;r_{55}\end{bmatrix}
\end{pmatrix}
\label{eq:template2-oe}
\end{equation}\]}

And the process errors would be

{\[\begin{equation}
\begin{bmatrix}
    w_{1} \\
    w_{2} \\
    w_{3} \end{bmatrix}_t
\sim \text{MVN} \begin{pmatrix}
 \begin{bmatrix}
    0 \\
    0 \\
    0 \end{bmatrix},
 \begin{bmatrix}
    q_{11}&amp;q_{12}&amp;q_{13}\\
    q_{12}&amp;q_{22}&amp;q_{23}\\
    q_{13}&amp;q_{23}&amp;q_{33}\end{bmatrix}
\end{pmatrix}.
\label{eq:template2-pe}
\end{equation}\]}

\hypertarget{sec-uss-3}{}
Section 3

Another section.

\hypertarget{sec-uss-discussion}{}
Discussion

Here is a discussion.

  \bibliography{tex/Fish507.bib,tex/book.bib,tex/packages.bib}

\end{document}
